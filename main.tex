% INSTITUTO FEDERAL DE EDUCAÇÃO, CIÊNCIA E TECNOLOGIA DE BRASÍLIA
%
% Template de Artigo Científico
% Conforme modelo do Manual de Normalização de Trabalhos Acadêmicos
%
% Desenvolvido por Lucas Santos de Oliveira
% E-mail: lucas.oliveira4@estudante.ifb.edu.br
%
% ===============================================================

\documentclass[a4paper, 12pt, oneside] {article}
\thispagestyle{empty}
% Pacotes Fundamentais, favor não alterar nada

% Configuração das margens

\usepackage[eft=2.5cm,top=2cm,right=2.5cm,bottom=2cm]{geometry}


\usepackage[utf8]{inputenc}
\usepackage[T1]{fontenc}
\usepackage[brazil]{babel}
\usepackage{hyphenat}


% Configuração dos parágrafos
\usepackage{indentfirst}
\setlength{\parindent}{1.25cm}

% Configuração do espaçamento
\usepackage{setspace}

% Configuração de alinhamento de texto
\usepackage{ragged2e}

% Configuração de numeração de página
\usepackage{fancyhdr}
\fancyhf{}
\rhead{\thepage}
\cfoot{}
\renewcommand{\headrulewidth}{0pt} % zera linha de cabeçalho
\pagestyle{fancy}
\setlength{\headheight}{15pt}

% Configuração da Fonte
%\usepackage{times} % Fonte: Times New Roman 
\usepackage{fontspec} % Pacote de Fontes
\setmainfont{calibri-regular.ttf}[Path=config/calibri/,BoldFont={calibri-bold.ttf}, ItalicFont={calibri-italic.ttf},BoldItalicFont={calibri-bold-italic.ttf}]


\usepackage[hang,flushmargin]{footmisc} % tirar o espaçamento da nota de rodapé

\usepackage{xcolor}

\makeatletter % para primeiras notas de rodapé
\def\@xfootnote[#1]{%
  \protected@xdef\@thefnmark{#1}%
  \@footnotemark\@footnotetext}
\makeatother

% Formatando Sections
\usepackage{mfirstuc}

\usepackage{titlesec}
\titleformat{\section}{\normalfont\fontsize{12}{12}\bfseries\MakeUppercase}{\thesection}{0.5em}{\tituloartigo}

\titleformat{\subsection}{\normalfont\fontsize{12}{12}\bfseries}{\thesubsection}{0.5em}{\capitalisewords}

\titleformat{\subsubsection}{\normalfont\fontsize{12}{12}\bfseries\itshape}{\thesubsubsection}{0.5em}{\capitalisewords}


%PACOTE DE CITAÇÕES
\usepackage[alf]{abntex2cite}

% Redefinindo a formatação para usar negrito em vez de itálico
\renewcommand{\emph}[1]{\textbf{#1}}

% Personalização do título das referências
\renewcommand{\refname}{Referências}

% Redefinindo formatação apenas para primeira letra em maiúscula
\newcommand{\ncite}[2][]{(\citeauthoronline{#2}, \citeyear{#2}\ifx&#1&\else, p. #1\fi)}

\usepackage{changepage}
\usepackage{ifoddpage}

% Definir ambiente quote para citações diretas com mais de 3 linhas
\newenvironment{myquote}{\par\fontsize{10pt}{12pt}\selectfont\begin{adjustwidth}{4cm}{0pt}}{\end{adjustwidth}\par}

% PACOTES ESSENCIAIS
\usepackage{graphicx}
\usepackage{float}
\usepackage{caption}
\usepackage{xurl}

% Define o tamanho da fonte para as legendas
\DeclareCaptionFont{mycaptionfont}{\fontsize{10pt}{12pt}\selectfont}

% Aplica o tamanho da fonte definido às legendas de figuras
\captionsetup{font=mycaptionfont}

% Redefine o separador das legendas para travessão (-)
\DeclareCaptionLabelSeparator{myhyphen}{ --- }
\captionsetup{font=mycaptionfont, labelsep=myhyphen}

% Define o espaçamento horizontal da nota de rodapé
\setlength{\footnotemargin}{10pt}
 % Não apague essa linha


% Preencha as informações do título, subtítulo, autor e orientador aqui
\newcommand {\tituloartigo}{[Template ]:}
\newcommand{\autor}{[Nome completo do(a) autor(a)]}
\newcommand{\emailautor}{email@provedor.do.aluno}
\newcommand{\orientador}{[Nome completo do(a) coorientador(a), (se houver)]}
\newcommand{\emailorientador}{email@provedor.do.orientador}
\newcommand{\subtituloingles}{[subtítulo (se houver) em língua estrangeira]}
\newcommand{\aprovacao}{Data de aprovação: XX/XX/XXXX}

% Inclua os pacotes necessários nas linhas abaixo
\usepackage{amsfonts,amsmath,amssymb}

\begin{document}
\singlespacing % espaçamento simples
\justifying % texto justificado

% Títulos e Subtítulos

% caso não tenha subtítulo, coloque % antes do \par \MakeLowercase{\subtitulo}
% Caso não queira utilizar um título em outro idioma, coloque % antes do \par \textbf{\MakeUppercase{\tituloingles}}
% Caso não tenha um subtítulo em outro idioma, coloque % antes do \par \MakeLowercase{\subtituloingles}


\begin{center}
\par\noindent\large\textbf{{\tituloartigo}}    \singlespacing
\end{center}

\begin{center}

 
    
    \par\noindent\textbf{\autor}
        \par \MakeLowercase{\emailautor}
\singlespacing

    \par\noindent\textbf{\orientador}
        \par \MakeLowercase{\emailorientador}

    \singlespacing

\end{center}


 
 % Insere os títulos e subtítulos

%%%%%%%%%%%%%%%%%%%%% RESUMO E ABSTRACT %%%%%%%%%%%%%%%%%%%%%%%%%%%%%%


\begin{center}

    \par {{Curso de pós graduação \textit{Lato Sensu} em Engenharia Elétrica}}
    \par {{IFG - Instituto Federal de Goiás}}
    
\end{center}

\noindent \textbf{Resumo}
\singlespacing

\noindent Este artigo tem como objetivo informar aos alunos do curso de pós-graduação \textit{lato sensu} como deve ser elaborado e apresentado o artigo científico que se constituirá de seu TCC – Trabalho de Conclusão de Curso. Nas seções do artigo são tratadas questões relativas à forma de apresentação do trabalho, bem como o que deve ser escrito em cada uma delas. Este documento encontra-se no modelo a ser seguido, então o aluno deve utilizá-lo como template. O resumo, redigido em língua portuguesa pelo próprio autor, deve trazer a síntese dos pontos relevantes do trabalho, tais como: tema, objeto da pesquisa, objetivos, materiais  e métodos utilizados, resultados alcançados e conclusões. O resumo não deve ultrapassar  250 palavras. No MS-Word pode-se utilizar o contador de palavras que se encontra na guia revisão, revisão de texto, contar palavras. O resumo deve ser digitado em um só parágrafo. As pessoas se baseiam no resumo para decidirem se irão ler ou não o restante do artigo. Assim, é importante que se resuma de maneira precisa e de forma atrativa os tópicos principais do artigo e as conclusões do trabalho. Deve-se escrever de forma bastante objetiva para evitar confusão na identificação da mensagem principal do artigo. No resumo não devem ser incluídas referências bibliográficas, citações diretas ou indiretas, figuras ou equações. Logo após o resumo devem ser apresentadas as palavras-chave do artigo. É importante que se escolham palavras-chave abrangentes, mas que ao mesmo tempo identifiquem os assuntos de que trata o artigo.\singlespacing

\noindent  Palavras-chave: palavra-chave 1; palavra-chave 2; palavra-chave 3. \bigskip


%%%%%%%%%%%%%%%%%%%%%%%%%%%%%% INTRODUÇÃO %%%%%%%%%%%%%%%%%%%%%%%%%%%%%

\noindent \textbf {1 Introdução}
\singlespacing

\par A introdução deve apresentar uma descrição geral do conteúdo do artigo científico sem entrar em muitos detalhes. Nesta parte do trabalho, apenas poucos parágrafos são o suficiente para sua apresentação. Recomenda-se uma página apenas. A introdução deve descrever brevemente a importância da área de estudo e do tema em foco e mostrar a relevância da publicação do artigo. Deve explicar como o trabalho pode contribuir para ampliar o conhecimento na área e se ele apresenta novos métodos para resolver ou abordar um problema. A introdução deve ser finalizada com a apresentação dos objetivos do trabalho. Deve-se evitar o uso de referências diretas e indiretas na introdução.

    \par O aluno do curso de pós-graduação deve fazer o seu TCC na forma de um artigo acadêmico-científico cujo formato está especificado neste documento. É importante que o aluno saiba que o artigo é aprovado somente pelo professor orientador. Uma vez aprovado pelo orientador, é que o aluno pode confirmar a participação da defesa na data informada pela equipe de TCC da universidade. Recomenda-se que o aluno leia o documento \textit{2-Regras e Orientações para a defesa do TCC e para a disciplina Metodologia da Pesquisa Científica EaD disponibilizado pelo orientador na plataforma AVA, na guia Materiais de Estudos da disciplina Metodologia}

\par \textbf{IMPORTANTE}: Este modelo/\textit{template} segue o NORMALIZA IFB 2ª Edição. 

\par Quaisquer dúvidas, acesse o \textcolor{blue}{\underline{NORMALIZA IFB 2ª Edição}} ou entre em contato com a Biblioteca do seu \textit{campus}.
\singlespacing

%%%%%%%%%%%%%%%%%%%%%%%%%%%%%% DESENVOLVIMENTO %%%%%%%%%%%%%%%%%%%%%%%

\noindent \textbf{2 Fundamentação teórica}

\par [OBRIGATÓRIO] Parte que contém a exposição ordenada e pormenorizada do assunto tratado. Deve ser dividido em seções e subseções, conforme a NBR 6024 (ABNT, 2012).
\singlespacing

\noindent \textbf{3 Materiais e Métodos}

\par [OBRIGATÓRIO] Parte que contém a exposição ordenada e pormenorizada do assunto tratado. Deve ser dividido em seções e subseções, conforme a NBR 6024 (ABNT, 2012).
\singlespacing

\noindent \textbf{4 Resultados ou discussão}

\par [OBRIGATÓRIO] Parte que contém a exposição ordenada e pormenorizada do assunto tratado. Deve ser dividido em seções e subseções, conforme a NBR 6024 (ABNT, 2012).
\singlespacing
%%%%%%%%%%%%%%%%%%%%%%%%%%%% CONSIDERAÇÕES FINAIS %%%%%%%%%%%%%%%%%%%%%%

\noindent \textbf{5 Conclusões}

\par [OBRIGATÓRIO] Parte final do artigo, na qual se apresentam as considerações correspondentes aos objetivos e/ou hipóteses.
\cite{oxley1992mt}
\singlespacing

\titleformat{\section}[block]{\normalfont\fontsize{12}{12}\bfseries\noindent}{\thesection}{1em}{\Uppercase}

\bibliography{bibliografia}

\singlespacing

\noindent \textbf{Agradecimentos}

\par Texto sucinto aprovado pelo periódico em que será publicado. Deve ser o último elemento pós-textual.
\singlespacing

\noindent\textbf{Anexo}

\par Texto ou documento não elaborado pelo autor, que serve de fundamentação, comprovação e ilustração.

\end{document}